\section*{SAMPLING THEORY}
\addcontentsline{toc}{section}{Sampling Theory}
\fancyhead[R]{SAMPLING THEORY}

\subsection*{Sampling Unit}
An element or a group of elements on which observations can be
taken is called a sampling unit.\\
The objective of the survey helps in determining the definition of
sampling unit.

\subsection*{Sampling Frame}
List of all the units of the population to be surveyed constitutes
the sampling frame.\\
All the sampling units in the sampling frame have identification
particulars.\\
For example, all the students in a particular university listed along
with their roll numbers constitutes the sampling frame.

\subsection*{Simple Random Sampling}
Simple random sampling (SRS) is a method of selection of a
sample comprising of n number of sampling units from the
population having N number of units such that every sampling
unit has an equal chance of being chosen.

\subsection*{Simple Random Sampling Without Replacement}
\subsubsection*{SRSWOR}
The sampling units are chosen without replacement in the sense
that the units once chosen are not placed back in the population.\\
SRSWOR is a method of selection of n units out of the N units one
by one such that at any stage of selection, any one of the
remaining units have the same chance of being selected, i.e. $\frac{1}{N}$.

\subsection*{Simple Random Sampling With Replacement}
\subsubsection*{SRSWR}
The sampling units are chosen with replacement in the sense that
the chosen units are placed back in the population.\\
SRSWR is a method of selection of n units out of the N units one by
one such that at each stage of selection, each unit has an equal
chance of being selected, i.e., $\frac{1}{N}$.

\subsection*{Sample Mean}
Population mean is generally measured by arithmetic mean (or
weighted arithmetic mean).\\
Let us consider the sample arithmetic mean $\bar{y}\ =\ \frac{1}{n}\sum_{i=1}^{n}y_i$ as an estimator of population mean $\bar{Y}\ =\ \frac{1}{N}\sum_{i=1}^{N}Y_i$\\
Estimate population mean $\bar{Y}\ =\ \frac{1}{N}\sum_{i=1}^{N}Y_i$ by sample mean $\bar{y}\ =\ \frac{1}{n}\sum_{i=1}^{n}y_i$\\
$\bar{y}$ is an unbiased estimator of $\bar{Y}$ under SRSWR and SRSwOR cases. 
\[ E(\bar{y})\ =\ \frac{1}{N}\sum_{i=1}^{N}y_i\ =\ \bar{Y}\]

\subsection*{Sample Varience}
Variance of sample mean under SRSWOR
\[ V(\bar{y_{WOR}})\ =\ E(\bar{y}\ -\ \bar{Y})^2\ =\ \frac{N\ -\ n}{Nn}S^2 \]
Variance of sample mean under SRSWR
\[ V(\bar{y_{WR}})\ =\ E(\bar{y}\ -\ \bar{Y})^2\ =\ \frac{N\ -\ 1}{Nn}S^2 \]

\subsection*{Standard Deviation of Sample Mean}
\[ \bar{y}\ =\ +\sqrt{Var(\bar{Y_{WOR}})}\ or\ +\sqrt{Var(\bar{Y_{WR}})} \]

\subsection*{Sampling for Proportions and Percentages}
In many situations, the characteristic under study on which the
observations are collected are qualitative in nature.
\subsubsection*{Sampling Procedure}
Follow the same sampling procedures used in case of \underline{quantitative} characteristics for drawing a sample for \underline{qualitative} characteristic.\\
SRSWOR and SRSWR procedures for drawing the samples remains
the same for qualitative and quantitative characteristics.

\subsubsection*{Mean}
Estimate population proportion by sample mean
\[ \bar{y}\ =\ p\ =\ \sum_{i=1}^{n}\frac{y_i}{n} \]
\subsubsection*{Variance}
The variance of p under SRSWOR and SRSWR are
\[ Var_{WOR}(p)\ =\ \frac{N-n}{N-1}\frac{PQ}{n} \]
\[ Var_{WR}(p)\ =\ \frac{PQ}{n} \]
respectively.

\subsection*{Stratified Sampling}
Important objective: Obtain an estimator of a population parameter
which can take care of all salient features of the population.\\
If the population is heterogeneous with respect to the
characteristic under study, then the sampling procedure used is
stratified sampling.

\subsection*{Bootstrap Methodology}
We are interested in finding the statistical properties of the
estimators such as the expressions for measures of accuracy.\\
Deriving the variance or standard error of these statistics is difficult.\\
Asymptotic theory can be used to derive them but they are not
available for small samples.\\
A modern alternative to the traditional approach is the
bootstrapping method.\\
Bootstrap is a powerful, computer‐based resampling method for
statistical inference without relying on too many assumption.\\
Bootstrap is a resampling method.\\
Simple random sampling with replacement (SRSWR) is used.\\
Samples are drawn independently by SRSWR from an existing
sample data of the same sample size n, and drawing inferences
from these resampled data.

\subsection*{Simple linear regression model}
Consider a simple linear regression model
\[ y\ =\ \beta_0\ +\ \beta_1X\ +\ \epsilon \]
y : Dependent or study variable\\
X : Independent or explanatory variable.\\
$\beta_0$ : Intercept term\\
$\beta_1$ : Slope parameter.\\
$\epsilon$ : Unobservable error component.It accounts for the failure of data to lie on the straight line and represents the difference between the true and observed realization of y.
\subsubsection*{OLSE}
The ordinary least squares estimates (OLSE) of $\beta_0$ and $\beta_1$ are denoted as $b_0$ and $b_1$ respectively.\\
$b_0\ =\ \bar{y}\ -\ b_1\bar{x}$\\
$b_1\ =\ \frac{s_{xy}}{s_{xx}}$\\
where\\
$s_{xy}\ =\ \sum_{i=1}^{n}(x_i\ -\ \bar{x})(y_i\ -\ \bar{y})$\\
$s_{xy}\ =\ \sum_{i=1}^{n}(x_i\ -\ \bar{x})^2$

\subsection*{Maximum Likelihood Property}
Assume $\epsilon_i`s$(i = 1,2,...n) are independent and identically distributed following a normal distribution N(0,$\sigma^2$).\\
The maximum likelihood estimates of the parameters $\beta_0$, $\beta_1$ and $\sigma^2$ of the linear regression model are
\begin{enumerate}
    \item $\bar{b_0}\ =\ \bar{y}\ -\ \bar{b_1}\bar{x}$ : same as OLSE
    \item $\bar{b_1}\ =\ \frac{s_{xy}}{s_{xx}}$ : same as OLSE
    \item $\bar{s^2}\ =\ \frac{\sum_{i=1}^{n}(y_i\ -\ \bar{b_0}\ -\ \bar{b_1}\bar{x}})$ : Different from OLSE
\end{enumerate}

\subsection*{Multiple linear regression model}
When we consider the regression modeling between the
dependent and more than one independent variables, then the
linear model is termed as the multiple linear regression model.\\
Let y denotes the dependent (or study) variable that is linearly
related to k independent (or explanatory) variables $X_1,X_2,...,X_k$ through the parameters $\beta_1,\beta_2,...,\beta_k$ and we write
\[ y\ =\ X_1\beta_1 \ +\ X_2\beta_2 \ +\ ,,, \ +\ X_k\beta_k\  +\ \epsilon \]
This is called as the multiple linear regression model.
\begin{itemize}
    \item The parameters $\beta_1,\beta_2,...,\beta_k$ are the regression coefficients associated with $X_1,X_2,...,X_k$ respectively and
    \item $\epsilon$ is the random error component reflecting the difference between the observed and fitted linear relationship.
\end{itemize}

\subsection*{Principle of OLSE}
The ordinary least squares (OLS) estimator of $\beta$ is
\[ b\ =\ (X`X)^{-1}X`y. \]
The estimator of $\sigma^2$ is
\[ \hat{\sigma^2}\ =\ \frac{(y-Xb)`(y-Xb)}{n-k} \]

\subsection*{Maximum Likelihood Estimation}
In the model $y\ =\ X\beta\ +\ \epsilon$, it is assumed that the errors are normally and independently distributed with constant variance i.e., $\epsilon\ ~\ N(0,\sigma^2I) $.\\
The likelihood function is the joint density of $\epsilon_1,\epsilon_2,...,\epsilon_n$.\\
Maximizing the log likelihood\\
The maximun likelihood estimators (mle) of $\beta$ and $\sigma^2$ are obtained by equating the first order derivatives of $ln\ L(\beta,\sigma^2)$ w.r.t $\beta$ and $\sigma^2$ to zero.\\
\textbf{Note:} OLSE and mle of $\beta$ are the same. So mle of $\beta$ is also an unbaised estimator of $\beta$ also
mle of $\sigma^2$ is biased estimator of $\sigma^2$.