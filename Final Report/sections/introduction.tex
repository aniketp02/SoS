\section*{Introduction}
\addcontentsline{toc}{section}{Introduction}
\fancyhead[R]{Introduction}

Data mining is a process of extracting and discovering patterns in large data sets involving methods at the intersection of machine learning, statistics, and database systems. Data mining is an interdisciplinary subfield of computer science and statistics with an overall goal to extract information (with intelligent methods) from a data set and transform the information into a comprehensible structure for further use. Data mining is the analysis step of the "knowledge discovery in databases" process, or KDD. Aside from the raw analysis step, it also involves database and data management aspects, data pre-processing, model and inference considerations, interestingness metrics, complexity considerations, post-processing of discovered structures, visualization, and online updating.\\
\\
The term "data mining" is a misnomer, because the goal is the extraction of patterns and knowledge from large amounts of data, not the extraction (mining) of data itself.\\
\\
While data analysis is used to test models and hypotheses on the datasets, data mining on the other hand uses machine learning and statistical models to uncover clandestine or hidden patterns in a large volume of data.\\
\\
In this report I try to understand some of the statistical and machine learning aspects of Data mining such as data management using SQL, data pre{-}processing, model and inference considerations and machine learning algorithms used in data mining.